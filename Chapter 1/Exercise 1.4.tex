\subsection{Exercise 4}



\includegraphics[scale=0.15]{Chapter 1/images/1_4.jpg}

The observer is at $O$, the quasar is at $Q$, shooting gas toward $A$.

Let $t_0$ be the time just as the gas started to eject. Then the observer at $O$, will see that at $t_0+D$ (where $c=1$ so $D$ is in units of time). After a time $T$, the jet has moved to point $A$. The observer will see that at $t_0+T+L$. Then the time difference is $\Delta t = (t_0+T+L)-(t_0+D)=T+L-D$. Using trigonometry we can find $L$:
\begin{align}
    L &= \sqrt{(D-Tv\cos{\theta})^2 + (Tv\sin{\theta})^2} \\
    &= D\sqrt{1-2\frac{Tv}{D}\cos{\theta} + \left(\frac{tV}{D}\right)^2} \\
    &\approx D - Tv\cos{\theta}
\end{align}
Where we used the fact that $Tv\ll D$. The time difference is then:
\begin{equation}
    \Delta t = T ( 1 - v\cos{\theta})
\end{equation}

The distance is $AB=Tv\sin{\theta}$, so the apparent velocity is:
\begin{equation}
    v_{app} = \frac{v\sin{\theta}}{1-v\cos{\theta}} 
\end{equation}
For $v=0.9,\theta=\frac{pi}{4}$, we get $v_{app}\approx1.75$.
