\subsection{Exercise 2}
Observer $A$, sees observer $B$ moving at velocity $v$. The time that it takes $B$ to move a full period ($L$) is $\tau_A=\frac{L}{v}$. The four-position of $B$ can be written as
\begin{equation}
    x_B^\mu = \begin{bmatrix}
        \lambda \\ x_0 + v \lambda \\y_0 \\ z_0
    \end{bmatrix}
\end{equation}
From equation 1.22 we can find the proper time
\begin{equation}
    \tau_B = \int_0^\frac{L}{v} \sqrt{1-v^2}d\lambda = \frac{L}{\gamma v}
\end{equation}
We see that there is a factor of $\gamma$ between the proper time of the observers. However, the situation is symmetric, in that from $B$'s frame, $A$ moves at velocity $-v$. Thus we expect the proper times to be the same. This is not consistent with our understanding of Lorentz invariance.