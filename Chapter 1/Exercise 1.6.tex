\subsection{Exercise 6}
\subsubsection{a}
At point $p$, we have $\lambda=1, \mu=0,\sigma=-1$.
\begin{align}
    \partial_i x^i(\lambda) &= (1, 2\lambda-2,-1) =_{\lambda=1} (1,0,-1) \\
    \partial_i x^i(\mu) &= (-\sin{\mu},\cos{\mu}, 1) =_{\mu=0} (0,1,1) \\
    \partial_i x^i(\sigma) &= (2\sigma, 3\sigma^2+2\sigma, 1) =_{\sigma=-1} (-2,1,1)
\end{align}
\subsubsection{b}
\begin{align}
    \frac{df}{d\lambda} &= \frac{d \lambda^2}{d\lambda} + \frac{d(\lambda-1)^4}{d\lambda} - \frac{d(-\lambda(\lambda-1)^2)}{d\lambda} \\
    &= 2\lambda +4(\lambda-1)^3 + 2\lambda (\lambda-1) +(\lambda-1)^2
\end{align}
and similarly for $\mu,\sigma$.