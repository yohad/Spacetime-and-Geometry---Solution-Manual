\subsection{Exercise 9}
We take "smoothing" to mean that $T_{\mu\nu}$  at a single point is equal to the average of $T_{\mu\nu}$ in a small volume around the point.
\begin{align}
    T_{\mu\nu} &= \frac{1}{\Delta V} \int_{\Delta V} T_{\mu\nu} d\boldsymbol{x} \\
    &= \frac{1}{\Delta V} \int_{\Delta V} \sum_a\frac{p_\mu^{(a)}p_\nu^{(a)}}{p^{0(a)}} \delta^{(3)} (\boldsymbol{x}-\boldsymbol{x}^{(a)})d\boldsymbol{x} \\
    &= \frac{1}{\Delta V} \sum_a \frac{p_\mu^{(a)}p_\nu^{(a)}}{p^{0(a)}} \int_{\Delta V} \delta^{(3)} (\boldsymbol{x}-\boldsymbol{x}^{(a)})d\boldsymbol{x} \\
    &= \frac{1}{\Delta V} \sum_a \frac{p_\mu^{(a)}p_\nu^{(a)}}{p^{0(a)}}
\end{align}
We can now start calculating the elements of the tensor.
\begin{equation}
    T_{00} = \frac{1}{\Delta V} \sum_a p^{(a)}_0
\end{equation}
$p_0$ is the energy of the particle. So $T_{00}=\rho$, the energy density.
\begin{equation}
    T_{i0} = \frac{1}{\Delta V} \sum_a p^{(a)}_i
\end{equation}
From the assumption that the velocities are isotropically distributed and that the sum is over enough particles (that the dust is dense in this region) we see that $T_{i0}=0$
\begin{equation}
    T_{ij} =  \frac{1}{\Delta V} \sum_a \frac{p_i^{(a)}p_j^{(a)}}{p^{0(a)}}
\end{equation}
Looking at a single particle in the sum, we also write $\Delta V=dx^idx^kdx^l$.
\begin{align}
    \frac{1}{\Delta V} \frac{p_ip_j}{p^{0}} &= \frac{1}{dx^idx^kdx^l} \frac{m\frac{dx^i}{d\tau} m\frac{dx^j}{d\tau}}{m\frac{dx^0}{d\tau}} \\
    &= \frac{dx^j}{dx^i} \frac{m}{dx^kdx^l} \frac{dx^i}{d\tau dx^0} \\
    &= \delta_i^j \frac{m}{dx^kdx^l} \frac{dx^i}{d\tau dx^0}
\end{align}
We can see that $T_{ii}$ is the pressure (check the units) on a small area $dx^kdx^l$. But we assumed no favoured direction, so the pressure is the same.
We can now write:
\begin{equation}
    T_{\mu\nu} = diag(\rho,p,p,p)
\end{equation}
which is the same as in 1.114.